\documentclass[11pt]{amsart}

\usepackage{amsmath,amsthm,amssymb,mathtools}
\usepackage{enumerate,enumitem}
\usepackage[utf8]{inputenc}
\usepackage[a4paper,margin=1in]{geometry}
\usepackage{hyperref}
\hypersetup{colorlinks=true,linkcolor=blue,citecolor=blue,urlcolor=blue}

\newtheorem{theorem}{Theorem}[section]
\newtheorem{lemma}[theorem]{Lemma}
\newtheorem{proposition}[theorem]{Proposition}
\newtheorem{corollary}[theorem]{Corollary}
\theoremstyle{definition}
\newtheorem{definition}[theorem]{Definition}
\newtheorem{remark}[theorem]{Remark}
\newtheorem{example}[theorem]{Example}

\newcommand{\N}{\mathbb{N}}
\newcommand{\Z}{\mathbb{Z}}
\newcommand{\C}{\mathbb{C}}
\newcommand{\R}{\mathbb{R}}
\newcommand{\Primes}{\mathcal{P}}

\title{Characterizing Arithmetic Through Dirichlet Series with Boolean Coefficients}

\author{Eduardo Zambrano}
\address{California Polytechnic State University, San Luis Obispo, CA 93407, USA}
\email{ezambran@calpoly.edu} 


\date{\today}

\begin{document}
\maketitle

\begin{abstract}
Could an alternative multiplication operation on $\N$ produce a Dirichlet series
resembling the Riemann zeta function? We show that any such operation must be
isomorphic to ordinary multiplication.

Given an operation $\star$ on $\N$ with unique factorization and Boolean multiplicative
coefficients $a(n) \in \{0,1\}$, we prove that if the Dirichlet series $Z(s) = \sum a(n)n^{-s}$ (i) admits an Euler product over the $\star$-irreducibles, (ii) has locally convergent factors, and (iii) saturates the boundary mass ($\limsup_{s\to 1^+}(s-1)Z(s) \ge 1$), then necessarily $a(n)=1$ for all $n$, $Z(s) = \zeta(s)$, and the monoid $(\N,\star)$ is isomorphic to $(\N, \cdot)$.

The proof is a short mass argument: since $a(n) \le 1$, we have $Z(s) \le \zeta(s)$, so any zero coefficient creates positive missing mass that contradicts boundary saturation. Unlike classical characterizations requiring functional equations and global analytic continuation, only an Euler product, local convergence, and boundary mass are needed. The three hypotheses are logically independent. The conclusion is sharp: we exhibit an operation $\star$ satisfying all hypotheses with $Z(s)=\zeta(s)$ and $a\equiv 1$ but $\star \neq \cdot$, showing that the conclusion cannot be upgraded from isomorphism to identity.
\end{abstract}

\textbf{Keywords:} Multiplicative functions, Dirichlet series, analytic rigidity, zeta function. \\
\textbf{MSC (2020):} Primary 11M41; Secondary 11A25, 11N37, 11M06.\\




\section{Introduction}

The Riemann zeta function $\zeta(s) = \sum_{n=1}^{\infty} n^{-s}$, convergent for $\Re(s) > 1$, encodes the multiplicative structure of the integers through its Euler product representation $\prod_{p \in \Primes} (1-p^{-s})^{-1}$, where $\Primes$ denotes the set of ordinary primes. This fundamental connection between analysis and arithmetic raises a natural question: could a different multiplication law on $\N$ produce a Dirichlet series with similar analytic properties? 

This paper proves that the answer is essentially \textbf{no}---provided the alternative multiplication satisfies mild regularity conditions. Under these conditions, the monoid $(\N, \star)$ must be isomorphic to $(\N, \cdot)$ and the Dirichlet series must equal $\zeta(s)$.

We consider Dirichlet series of the form
\[
Z(s) = \sum_{n \geq 1} a(n)n^{-s}
\]
where the coefficients $a(n) \in \{0,1\}$ are multiplicative with respect to some associative, commutative operation $\star$ on $\N$ admitting unique factorization into $\star$-irreducibles $P$. We ask: what constraints does the analytic behavior of $Z(s)$ impose on the multiplication $\star$?

Our main result shows that three minimal conditions suffice:
\begin{itemize}
\item \textbf{Euler Product:} $Z(s) = \prod_{p \in P} F_p(s)$, where $F_p(s) := 1 + \sum_{k \geq 1} a(p^{\star k})(p^{\star k})^{-s}$.
\item \textbf{Local Convergence:} Each local Euler factor has a finite limit: $F_p(1) := \lim_{s\to 1^+} F_p(s) < \infty$ for every $\star$-irreducible $p$.
\item \textbf{Boundary Mass Saturation:} The series diverges at the critical rate as $s \to 1^+$, i.e., $\limsup_{s\to1^+}(s-1)Z(s) \ge 1$.
\end{itemize}

\begin{center}
\fbox{\begin{minipage}{0.9\textwidth}
\textbf{Main Theorem (Theorem~\ref{thm:main}):} Let $(\star, A)$ be a Boolean multiplicative monoid (Definition~\ref{def:bmd}). If
\begin{enumerate}[label=\textbf{(E\arabic*)},leftmargin=2.5em,topsep=0.3em,itemsep=0.1em]
\item $Z(s) = \prod_{p\in P} F_p(s)$ for $\Re s > 1$ (Euler product),
\item $F_p(1) := \lim_{s\to 1^+} F_p(s) < \infty$ for every $\star$-irreducible $p$
      (local convergence), and
\item $\limsup_{s\to1^+}(s-1)Z(s) \ge 1$ (boundary mass saturation),
\end{enumerate}
then $a(n)=1$ for all $n$, $Z(s) = \zeta(s)$, and $(\N,\star) \cong (\N,\cdot)$ as monoids.
\end{minipage}}
\end{center}

\subsection{Relationship to Prior Work}

Our characterization of multiplication through Dirichlet series connects to 
several research programs:

\paragraph{Beurling's generalized primes.}
Beurling \cite{Beurling} initiated the study of generalized number systems 
where "integers" are generated by "primes" through multiplication, asking when 
such systems satisfy asymptotic laws (prime number theorem, divisor function 
growth, etc.). Diamond \cite{Diamond1,Diamond2} developed this program further, 
establishing conditions under which the counting function $N(x)$ determines 
analytic properties of the associated zeta function. Beurling's framework fixes ordinary multiplication but 
varies the primes (and thus the integers). We fix the integers $\N$ but vary 
the multiplication operation $\star$. This is an orthogonal direction of 
generalization.

Knopfmacher \cite{Knopfmacher} developed an abstract theory of semigroups with
norm functions admitting zeta functions $\zeta_{\mathcal{A}}(s) = \sum_{a \in \mathcal{A}} N(a)^{-s}$.
He established analogues of classical number-theoretic results (Euler product,
prime number theorem) under various axioms on the semigroup structure. Knopfmacher studies \emph{which semigroups produce
zeta-like functions}. We study the \emph{inverse problem}: given that a
Dirichlet series resembles $\zeta(s)$, what does this force about the underlying
operation? Our Boolean framework and minimal hypotheses provide sharp rigidity
results complementing Knopfmacher's axiomatic approach.

\paragraph{Analytic characterizations of $\zeta(s)$.}
Classical characterizations of $\zeta(s)$ operate on the analytic side: Hamburger \cite{Hamburger} showed that the Riemann functional equation determines $\zeta(s)$ among Dirichlet series of moderate growth, and Hecke \cite{Hecke} extended this framework to $L$-functions. The Selberg class \cite{Selberg} axiomatizes these analytic properties (see \cite{Tenenbaum} for background on multiplicative number theory and Dirichlet series). By contrast, the present work characterizes the \emph{multiplicative structure} underlying $\zeta(s)$, asking not which Dirichlet series satisfy a functional equation, but which multiplication on $\N$ can produce $\zeta(s)$ at all.

To our knowledge, the specific question---``what minimal analytic properties of
$Z(s)$ uniquely determine the multiplication operation $\star$ on $\N$?''---has
not been addressed in this form. Our contribution is fourfold:

\begin{enumerate}
\item \textbf{Minimal analytic hypotheses:} A Boolean multiplicative monoid (Definition~\ref{def:bmd}) is a purely algebraic object (monoid with Boolean multiplicative function).  The three analytic conditions (E1)--(E3)---Euler product, local convergence of Euler factors, and boundary mass saturation---are stated separately as hypotheses, placing them on equal footing.  The Euler product (E1) is the standard requirement that $Z(s)$ respect its $\star$-prime structure.  No functional equation, continuation, upper growth bounds, or comparison to ordinary prime powers is required.
\item \textbf{Mass argument:} The proof is a short inequality: since $a(n) \le 1$, any zero coefficient creates missing mass that contradicts boundary saturation. This avoids Tauberian methods and complex integration.
\item \textbf{Sharp conclusion:} We prove $(\N,\star) \cong (\N,\cdot)$ and exhibit a counterexample showing that the conclusion cannot be upgraded to $\star = \cdot$.
\item \textbf{Analytic vs.\ algebraic determination:} The analytic conditions
      (E1)--(E3) determine the multiplicative monoid structure uniquely
      (up to isomorphism) but cannot determine its concrete realization on
      $\N$. Upgrading from $\cong$ to $=$ requires the non-analytic condition
      of order-compatibility (Corollary~\ref{cor:order-compat}).
\end{enumerate}

The rest of the paper is organized as follows: Section~\ref{sec:defs} provides definitions and motivating examples. Section~\ref{sec:hypotheses} states our three analytic hypotheses (E1)--(E3). Section~\ref{sec:norm-compat} derives norm compatibility from the Euler product. Section~\ref{sec:mass} presents the mass argument. Section~\ref{sec:main} proves the main theorem. Section~\ref{sec:sharpness} demonstrates logical independence of the hypotheses. Section~\ref{sec:concluding} offers concluding remarks.



\section{Definitions and Motivating Examples}\label{sec:defs}

We recall standard terminology from factorization theory; see \cite{GHK} for a comprehensive treatment.

\begin{definition}[Monoid terminology]\label{def:monoid}
Let $(M,\cdot\,,1)$ be a commutative monoid with identity~$1$.
\begin{enumerate}[label=(\roman*)]
    \item An element $u \in M$ is a \textbf{unit} if $uv=1$ for some $v \in M$.
          $M$ is \textbf{reduced} if $1$ is the only unit.
    \item A non-unit $a \in M$ is an \textbf{atom} (or \textbf{irreducible})
          if $a=bc$ implies $\{b,c\}=\{1,a\}$.
    \item $M$ is \textbf{factorial} (or a \textbf{unique factorization monoid})
          if every non-unit has a unique factorization into atoms, up to order.
\end{enumerate}
\end{definition}

\begin{definition}[Boolean multiplicative monoid]\label{def:bmd}
A \textbf{Boolean multiplicative monoid} is a pair $(\star, A)$ where:
\begin{enumerate}[label=(\roman*)]
    \item $(\N,\star,1)$ is a reduced factorial commutative monoid
          (Definition~\ref{def:monoid}).
          Its atoms are called \textbf{$\star$-irreducibles}; we denote their
          set by~$P$.  Every $n \ge 2$ has a unique factorization
          $n = p_1^{\star k_1}\star\cdots\star p_r^{\star k_r}$ with $p_i \in P$.
          Cancellation---if $a\star b=a\star c$ then $b=c$---follows from
          unique factorization.
    \item $A \subseteq \N$ with $1 \in A$, and its characteristic function
          $a(n) := \mathbf{1}_{n \in A} \in \{0,1\}$ is
          \textbf{$\star$-multiplicative}: if $n$ and $m$ are $\star$-coprime
          (share no common $\star$-irreducible factors), then
          $a(n \star m) = a(n)a(m)$.
\end{enumerate}
\end{definition}

Given a pair $(\star, A)$ as above, we define the associated \emph{Dirichlet series} and \emph{local Euler factors}:
\[
Z(s) \;:=\; \sum_{n\ge1} a(n)\,n^{-s}, \qquad
F_p(s) \;:=\; 1 + \sum_{k\ge1} a(p^{\star k})(p^{\star k})^{-s},
\qquad \Re s > 1.
\]
Since $a(n)\le 1$, we have $Z(s)\le\zeta(s)<\infty$ for $\Re s>1$ (see~\cite{Tenenbaum} for standard convergence criteria).

\subsection{Examples}

\begin{example}[Classical multiplication]\label{ex:classical}
Let $\star$ be ordinary multiplication and $A=\N$. Then:
\begin{itemize}
\item The $\star$-irreducibles coincide with the ordinary primes: $P=\Primes=\{2,3,5,7,\ldots\}$.
\item $a(n)=1$ for all $n\ge1$, trivially multiplicative.
\item $Z(s)=\sum_{n=1}^{\infty}n^{-s}=\zeta(s)$.
\end{itemize}
\end{example}

\begin{example}[Square-free support]\label{ex:squarefree}
With ordinary multiplication, let $A$ be the set of square-free positive integers. Then $a(n)=\mu^2(n)$ and
\[
Z(s)=\sum_{n=1}^{\infty}\mu^2(n)n^{-s}
     =\frac{\zeta(s)}{\zeta(2s)}
     =\prod_{p\in\Primes}(1+p^{-s}).
\]
This series has a simple pole at $s=1$ with residue $6/\pi^2\ne1$, showing how missing prime powers reduce boundary mass.
\end{example}

\begin{example}[Sparse support]\label{ex:sparse}
Let $A=\{2^k:k\ge0\}$ with ordinary multiplication. Then
\[
Z(s)=\sum_{k=0}^{\infty}2^{-ks}=\frac{1}{1-2^{-s}}
\]
is meromorphic on $\C$ with simple poles at $s=2\pi in/\log2$ for $n\in\Z$.  
Since $Z(1)=2$ is finite, $(s-1)Z(s)\to0$ as $s\to1^+$, so $\limsup_{s\to1^+}(s-1)Z(s)=0<1$.
This violates (E3), illustrating analytic ``thinness'' at the boundary.
\end{example}

\medskip
\noindent

\subsection*{When the Euler product fails}
The following examples illustrate how arithmetic coherence breaks down once either unique factorization or multiplicativity is lost (and why the Boolean constraint matters for identifiability).

\begin{example}[No Euler product I: failure of unique factorization]\label{ex:nonunique}
Let $S=\{1\}\cup\{n \ge 4: n \text{ is composite}\}=\{1,4,6,8,9,10,12,14,15,\ldots\}$ under ordinary multiplication.
This set is closed under multiplication, and its irreducibles are the composite numbers that cannot be factored into smaller composites (e.g.\ $4,6,8,9,10,14,15,\ldots$). However, factorization is not unique since $36=6\cdot6=4\cdot9$.
Without unique factorization there is no well-defined Euler product, and such structures lie outside our Boolean multiplicative framework.
\end{example}

\begin{example}[No Euler product II: failure of multiplicativity]\label{ex:noneuler}
Consider
\[
F(s)=\zeta(s)-2^{-s}
=\sum_{n\ge1} b(n)n^{-s},\qquad
b(1)=1,\ b(2)=0,\ b(n)=1\ \ (n\ge3).
\]
Here $b$ is not multiplicative: since $2$ and $3$ are coprime, multiplicativity would require $b(6)=b(2)b(3)=0\cdot1=0$, but $b(6)=1$. Thus $F$ admits no Euler product and cannot arise from any $\star$ with unique factorization.  
This shows that multiplicativity is indispensable for linking local data to global analytic behavior.
\end{example}

\begin{example}[Why Boolean coefficients matter]\label{ex:nonboolean}
If coefficients are allowed beyond $\{0,1\}$, identifiability from $Z(s)$ alone can fail.  
For any nontrivial real character $\chi$ with $\chi^2=1$,
\[
\sum_{n\ge1}\frac{1+\chi(n)}{n^s}=\zeta(s)+L(s,\chi)
\]
has coefficients in $\{0,1,2\}$, and different $\chi$ give different coefficient patterns despite sharing the same simple analytic building blocks.  
The Boolean restriction restores uniqueness at the level of the Dirichlet series.
\end{example}

\section{Analytic Rigidity Conditions}\label{sec:hypotheses}

We state three analytic hypotheses on a Boolean multiplicative monoid $(\star,A)$ and its associated Dirichlet series $Z(s)=\sum_{n\ge1} a(n)n^{-s}$.

\begin{enumerate}[label=\textbf{(E1)},leftmargin=2.2em]
  \item \textbf{Euler Product.}
  The Dirichlet series factors over the $\star$-irreducibles:
  \begin{equation}\label{eq:E1}
     Z(s) \;=\; \prod_{p\in P} F_p(s), \qquad \Re s > 1.
  \end{equation}
\end{enumerate}

\begin{enumerate}[label=\textbf{(E2)},leftmargin=2.2em]
  \item \textbf{Local Convergence.}
  For every $\star$-irreducible $p$, the local Euler factor has a finite limit:
  \begin{equation}\label{eq:E2}
     F_p(1) \;:=\; \lim_{s\to1^+} F_p(s) \;<\; \infty.
  \end{equation}
\end{enumerate}

\begin{enumerate}[label=\textbf{(E3)},leftmargin=2.2em]
  \item \textbf{Boundary Mass Saturation.}
  \begin{equation}\label{eq:E3}
     \limsup_{s\to1^+} (s-1)Z(s) \;\ge\; 1.
  \end{equation}
\end{enumerate}

\subsection{Remarks on the Hypotheses}

\paragraph{On (E1): Euler Product.}
The Euler product is the standard requirement that $Z(s)$ respect its $\star$-prime structure.
It converges absolutely for $\Re s>1$ since $Z(s)\le\zeta(s)<\infty$.

\emph{Why this is natural.}
The Euler product expresses multiplicative independence analytically: each $\star$-irreducible contributes to $Z(s)$ through its own local factor $F_p(s)$, and these contributions combine independently.

\paragraph{On (E2): Local Convergence.}

\emph{What it says.} Each local Euler factor $F_p(s)$ remains bounded as $s\to 1^+$. Equivalently, the series $\sum_{k\ge1} a(p^{\star k})/(p^{\star k})$ converges for every $\star$-irreducible $p$.

\emph{Why this is natural.}
Local convergence requires that the $\star$-powers $p^{\star k}$ grow fast enough that their reciprocals are summable. This excludes degenerate operations where powers collapse or grow too slowly (see Examples~\ref{ex:violate-E2-diverge} and~\ref{ex:violate-E2-subexp}).


\paragraph{On (E3): Boundary Mass Saturation.}

\emph{What it says.} Since $a(n) \le 1$ gives $Z(s) \le \zeta(s)$ termwise,
\[
\limsup_{s\to 1^+}(s-1)\,Z(s)\;\le\; \lim_{s\to 1^+}(s-1)\,\zeta(s) = 1.
\]
We call $\limsup_{s\to 1^+}(s-1)Z(s)$ the \emph{boundary mass} of the series.
Condition~(E3) requires that this upper bound be achieved.

\emph{Why this is natural.}
The boundary mass measures the density of the support~$A$: roughly,
if $A$ has natural density~$d$ then the boundary mass equals~$d$.
Thus (E3) says the support is as dense as Boolean coefficients permit---no
$\star$-irreducibles or powers are missing in aggregate
(see Examples~\ref{ex:violate-E3-missing} and~\ref{ex:squarefree}).
No meromorphic continuation or precise residue is required---only that
the density achieves the critical threshold.
The Euler product~(E1) then upgrades this aggregate condition to a pointwise one:
any single zero coefficient $a(p^{\star j})=0$ would propagate through the
multiplicative structure, creating enough missing mass to pull the boundary
mass strictly below~$1$.



\section{Proof Strategy}\label{sec:roadmap}

The proof proceeds in four steps:

\begin{enumerate}
\item \textbf{Step 1: Norm compatibility (Lemma~\ref{lem:norm-compat}).}

The Euler product (E1) forces $m\star n = m\cdot n$ for $\star$-coprime $m,n$, so the only freedom in $\star$ is how iterated powers of a single irreducible are defined (Section~\ref{sec:norm-compat}).

\item \textbf{Step 2: $Z \le \zeta$ (trivial).}

Since $a(n) \le 1$ for all $n$, we have $Z(s) = \sum a(n)n^{-s} \le \sum n^{-s} = \zeta(s)$.

\item \textbf{Step 3: Mass argument $\Rightarrow$ $a \equiv 1$ (Proposition~\ref{prop:mass}).}

Suppose $a(p^{\star j}) = 0$ for some $\star$-irreducible $p$ and $j\ge 1$. The Euler product (E1) gives $Z = F_p \cdot Z'_p$, separating the $p$-part from the coprime part. Every integer of the form $p^{\star j} \cdot m$ with $m$ coprime to $p$ and $a(m)=1$ satisfies $a(p^{\star j}\cdot m)=0$, contributing missing mass $(p^{\star j})^{-s}\cdot Z'_p(s)$ to $\zeta(s)-Z(s)$. Multiplying by $(s-1)$ and using (E2)+(E3) yields a contradiction.

\item \textbf{Step 4: Free monoid isomorphism $\Rightarrow$ $\star \cong \cdot$ (Theorem~\ref{thm:main}).}

Since $a \equiv 1$, both $(\N,\star)$ and $(\N,\cdot)$ are free commutative monoids on countably many generators. Any bijection between the $\star$-irreducibles $P$ and the ordinary primes $\Primes$ extends to a monoid isomorphism.
\end{enumerate}



\section{Norm Compatibility}\label{sec:norm-compat}

The Euler product hypothesis (E1) alone  has a fundamental rigidity consequence: it ties $\star$-multiplication of coprime inputs to ordinary multiplication.

\begin{lemma}[Norm compatibility]\label{lem:norm-compat}
The Euler product \textup{(E1)} implies that $\star$-multiplication of coprime inputs coincides with ordinary multiplication:
\begin{equation}\label{eq:norm-compat}
m\star n \;=\; m\cdot n \qquad\text{whenever $m$ and $n$ are $\star$-coprime.}
\end{equation}
\end{lemma}

\begin{proof}
Expanding the product $\prod_{p\in P}F_p(s)$, each term is obtained by choosing one summand from finitely many factors and taking the product of the remaining $1$'s.  A typical nonzero term selects exponents $k_1,\dots,k_r\ge1$ at distinct $\star$-irreducibles $p_1,\dots,p_r$ (with $a(p_i^{\star k_i})=1$) and contributes
\[
\prod_{i=1}^r (p_i^{\star k_i})^{-s}
\;=\;\Bigl(\prod_{i=1}^r p_i^{\star k_i}\Bigr)^{\!-s}
\]
to the product.  The Euler product identity equates this expanded sum with $Z(s)=\sum a(n)n^{-s}$, so by uniqueness of Dirichlet series with non-negative coefficients (see~\cite[Theorem~11.1]{Tenenbaum}), each $\star$-factorization $n=p_1^{\star k_1}\star\cdots\star p_r^{\star k_r}$ must satisfy
\begin{equation}\label{eq:norm-match}
n \;=\; p_1^{\star k_1}\cdot\,\cdots\,\cdot p_r^{\star k_r}.
\end{equation}
In particular, for $\star$-coprime $m=p_1^{\star k_1}\star\cdots\star p_r^{\star k_r}$ and $n=q_1^{\star \ell_1}\star\cdots\star q_t^{\star \ell_t}$ (with all $p_i,q_j$ distinct), applying \eqref{eq:norm-match} to $m\star n$ gives $m\star n=\prod p_i^{\star k_i}\cdot\prod q_j^{\star\ell_j}=m\cdot n$.
\end{proof}

The consequence of Lemma~\ref{lem:norm-compat} is that the only freedom in $\star$ is how iterated powers of a single $\star$-irreducible are defined: once the $\star$-power sequences $\{p^{\star k}\}_{k\ge0}$ are specified for each $p\in P$, the operation on arbitrary elements is determined by unique factorization and \eqref{eq:norm-compat}.

Conversely, norm compatibility implies the Euler product: if $m \star n = m \cdot n$ for $\star$-coprime inputs, then the standard expansion of $\prod_{p \in P} F_p(s)$ over finitely supported selections reproduces $\sum a(n) n^{-s}$ termwise, since each $\star$-factorization $n = p_1^{\star k_1} \star \cdots \star p_r^{\star k_r}$ contributes $(p_1^{\star k_1} \cdots p_r^{\star k_r})^{-s} = n^{-s}$ by norm compatibility. Thus (E1) and \eqref{eq:norm-compat} are equivalent characterizations of the same structural constraint; the Lean formalization axiomatizes (E1) in its norm-compatibility form (see Section~\ref{sec:concluding}).

\begin{example}[Twisted multiplication]\label{ex:twisted}
Let $\phi:\N\to\N$ swap $2\leftrightarrow3$ while fixing all other integers, and define
$n\star m=\phi^{-1}(\phi(n)\cdot\phi(m))$. Then:
\begin{itemize}
\item $2$ and $3$ are both $\star$-irreducible (since $\phi(2)=3$ and $\phi(3)=2$ are prime);
\item $3\star3=\phi^{-1}(2\cdot2)=4$, so $3^{\star2}=4<9=3^2$.
\end{itemize}
This deformation preserves unique factorization and multiplicativity but produces ``flattened'' powers: $3^{\star 2} = 4 < 9 = 3^2$.  It violates (E1): since $2\star 5=\phi^{-1}(\phi(2)\cdot\phi(5))=\phi^{-1}(15)=15\ne 10=2\cdot 5$, the Euler product $\prod F_p(s)$ does not equal $\sum a(n)n^{-s}$.
\end{example}


\section{The Mass Argument}\label{sec:mass}

The key observation is that $a(n) \le 1$ gives $Z(s) \le \zeta(s)$ with no hypothesis on $\star$. Any zero coefficient then creates positive missing mass, which contradicts boundary saturation.

\begin{proposition}[Full support]\label{prop:mass}
Let $(\star, A)$ be a Boolean multiplicative monoid. If \textbf{(E1)}, \textbf{(E2)}, and \textbf{(E3)} hold, then
\begin{equation}
a(p^{\star k})=1 \quad \text{for every } \star\text{-irreducible } p\in P \text{ and every } k \ge 1.
\end{equation}
Consequently, $a(n)=1$ for all $n\ge 1$ and $Z(s)=\zeta(s)$ for $\Re s > 1$.
\end{proposition}

\begin{proof}
\textbf{Step 1: $Z \le \zeta$.} Since $a(n) \in \{0,1\}$, we have $a(n)\le 1$ for every $n$, so
\begin{equation}\label{eq:zeta_over_Z}
Z(s)=\sum_{n\ge 1} a(n)n^{-s}\le \sum_{n\ge 1} n^{-s}=\zeta(s) \qquad (s>1).
\end{equation}

\textbf{Step 2: Vanishing gap.} From \eqref{eq:zeta_over_Z} and $\lim_{s\to1^+}(s-1)\zeta(s)=1$, (E3) provides a sequence $s_n\downarrow 1$ with $(s_n-1)Z(s_n)\to L\ge 1$. Then
\[
(s_n-1)\bigl(\zeta(s_n)-Z(s_n)\bigr) = (s_n-1)\zeta(s_n)-(s_n-1)Z(s_n)\to 1-L \le 0.
\]
Since $\zeta-Z\ge 0$, this forces
\begin{equation}\label{eq:gap}
(s_n-1)\bigl(\zeta(s_n)-Z(s_n)\bigr)\to 0.
\end{equation}

\textbf{Step 3: Any zero creates positive missing mass.} Suppose, for contradiction, that $a(p^{\star j})=0$ for some $\star$-irreducible $p$ and some $j\ge 1$. By (E1), write $Z(s)=F_p(s)\cdot Z'_p(s)$, where
\[
Z'_p(s) := \prod_{\substack{q\in P\\ q\ne p}} F_q(s) = \sum_{\substack{n\ge 1\\ n\ \star\text{-coprime to } p}} a(n)\,n^{-s}.
\]
For every $n$ that is $\star$-coprime to $p$ with $a(n)=1$, the integer $m = p^{\star j}\star n = p^{\star j}\cdot n$ (by norm compatibility, Lemma~\ref{lem:norm-compat}) satisfies $a(m)=a(p^{\star j})\cdot a(n)=0$ by $\star$-multiplicativity. These integers $p^{\star j}\cdot n$ are distinct (multiplication by $p^{\star j}\ge 2$ is injective on $\N$) and each satisfies $a(p^{\star j}\cdot n)=0$, so they contribute to the gap:
\begin{equation}\label{eq:missing}
\zeta(s)-Z(s) \ge \sum_{\substack{n\ge 1,\; n\ \star\text{-coprime to }p\\ a(n)=1}} (p^{\star j}\cdot n)^{-s} = (p^{\star j})^{-s}\cdot Z'_p(s) = \frac{(p^{\star j})^{-s}\cdot Z(s)}{F_p(s)}.
\end{equation}

\textbf{Step 4: Contradiction.} Multiply \eqref{eq:missing} by $(s_n-1)$:
\[
(s_n-1)\bigl(\zeta(s_n)-Z(s_n)\bigr) \ge (p^{\star j})^{-s_n}\cdot\frac{(s_n-1)Z(s_n)}{F_p(s_n)}.
\]
As $n\to\infty$: the left side $\to 0$ by \eqref{eq:gap}; the right side $\to (p^{\star j})^{-1}\cdot L/F_p(1)>0$, since $p^{\star j}\ge 2$, $L\ge 1$, and $F_p(1)<\infty$ by (E2). This is a contradiction.

\textbf{Step 5: Full support and $Z=\zeta$.} Therefore $a(p^{\star k})=1$ for all $p\in P$ and $k\ge 1$. By unique factorization, every $n\ge 2$ has a representation $n=p_1^{\star \alpha_1}\star\cdots\star p_r^{\star \alpha_r}$, and $\star$-multiplicativity gives $a(n)=\prod a(p_i^{\star \alpha_i})=1$. Hence $a\equiv 1$ and $Z(s)=\sum n^{-s}=\zeta(s)$.
\end{proof}

\section{Complete Rigidity: The Main Theorem}\label{sec:main}

\begin{theorem}[Complete Rigidity---Main Result]\label{thm:main}
Let $(\star, A)$ be a Boolean multiplicative monoid. If \textbf{(E1)}, \textbf{(E2)}, and \textbf{(E3)} hold, then necessarily:
\begin{enumerate}
  \item $a(n)=1$ for all $n\ge1$,
  \item $Z(s)=\zeta(s)$ for $\Re s>1$,
  \item the monoid $(\N, \star)$ is isomorphic to $(\N, \cdot)$.
\end{enumerate}
\end{theorem}

\begin{proof}
Conclusions (1) and (2) are immediate from Proposition~\ref{prop:mass}.

For (3): Since $a\equiv 1$, the monoid $(\N,\star)$ is a free commutative monoid on the generators $P$ (the $\star$-irreducibles), and $(\N,\cdot)$ is a free commutative monoid on the ordinary primes $\Primes$. Both generator sets are countably infinite: $P$ is infinite because if $P$ were finite then $Z(s) = \prod_{p\in P} F_p(s)$ would be a finite product of factors each finite at $s=1$ (by (E2)), hence finite at $s=1$, contradicting the pole of $Z(s)=\zeta(s)$; and $\Primes$ is infinite by Euclid's theorem. Any bijection $\phi: P \to \Primes$ extends uniquely to a monoid isomorphism $\Phi: (\N,\star) \to (\N,\cdot)$ by
\[
\Phi\bigl(p_1^{\star \alpha_1}\star\cdots\star p_r^{\star \alpha_r}\bigr) = \phi(p_1)^{\alpha_1}\cdots\phi(p_r)^{\alpha_r}.
\]
This is well-defined and bijective by unique factorization in both monoids, and preserves the monoid operation by construction.
\end{proof}

\begin{remark}[The isomorphism need not be the identity]\label{rem:psi-twist}
Theorem~\ref{thm:main} concludes $(\N, \star) \cong (\N, \cdot)$, but not $\star = \cdot$.  To see this concretely, let the $\star$-irreducibles be the ordinary primes $P=\Primes$, with $q^{\star k}=q^k$ for every prime $q\ge 3$, but with the second and third $\star$-powers of $2$ swapped:
\[
2^{\star 2}=8, \qquad 2^{\star 3}=4, \qquad 2^{\star k}=2^k \text{ for } k\ne 2,3.
\]
For $\star$-coprime elements, $\star$ equals ordinary multiplication by Lemma~\ref{lem:norm-compat}.  Let $A=\N$, so $a\equiv1$.  Every $n\ge 1$ has a unique $\star$-factorization (the $2$-adic valuation $v$ of $n$ determines the $\star$-exponent of $2$ via the permuted correspondence, while the odd part factors as usual), giving $Z(s)=\zeta(s)$.  All three hypotheses hold: (E1) because coprime inputs multiply ordinarily, (E2) because $F_2(1)=1+\sum_k (2^{\star k})^{-1}$ is a rearrangement of $1+\sum_k 2^{-k}<\infty$, and (E3) because $(s-1)Z(s)\to 1$.

Yet $\star\ne\cdot$ since $2\star 2=2^{\star 2}=8\ne 4=2\cdot 2$.  The isomorphism $\phi$ that ``unswaps'' the tower (sending $4\leftrightarrow 8$, $12\leftrightarrow 24$, etc.)\ maps $(\N,\star)$ onto $(\N,\cdot)$, but $\phi\ne\mathrm{id}$. Any bijection between $P$ and $\Primes$ gives a different isomorphism; the theorem guarantees at least one exists, but does not single out a canonical choice.
\end{remark}

The analytic conditions (E1)--(E3) determine the monoid structure completely, but cannot determine its concrete realization on~$\N$: the local Euler factor $F_p(s)=1+\sum_{k\ge 1}(p^{\star k})^{-s}$ is invariant under any permutation of the tower $\{p^{\star k}\}_{k\ge 1}$, so no condition on $Z(s)=\prod F_p(s)$ can distinguish the identity ordering from a rearrangement (Remark~\ref{rem:psi-twist}).  Upgrading $\cong$ to $=$ requires a non-analytic input.

\begin{corollary}[From isomorphism to identity]\label{cor:order-compat}
Under \textbf{(E1)}--\textbf{(E3)}, the operation $\star$ equals ordinary multiplication if and only if\/ $\star$ is \textbf{order-compatible}: $a\le b$ implies $a\star c\le b\star c$ for all $a,b,c\in\N$.
\end{corollary}

\begin{proof}
Necessity is immediate: ordinary multiplication is order-compatible.

For sufficiency, order-compatibility implies \emph{tower monotonicity}: $p^{\star k} < p^{\star(k+1)}$ for every irreducible~$p$ and $k\ge 1$ (from $1<p$ and cancellation).  Since $a\equiv 1$ by Theorem~\ref{thm:main}, matching local Euler factors in $\prod_{p\in P}F_p(s)=\prod_{q\in\Primes}(1-q^{-s})^{-1}$ (by uniqueness of Dirichlet series with non-negative coefficients; see~\cite[Theorem~11.1]{Tenenbaum}) gives, for each irreducible~$p$, a prime~$q$ with $\{p^{\star k}:k\ge 1\}=\{q^k:k\ge 1\}$ as sets.  Tower monotonicity forces $p^{\star k}=q^k$ for all~$k$; taking $k=1$ gives $p=q$, hence $\star=\cdot$.
\end{proof}

\begin{remark}[Beyond Boolean coefficients]\label{rem:beyond-boolean}
The mass argument extends to $a(n)\in[0,1]$ (still $\star$-multiplicative): since $a(n)\le 1$ still gives $Z\le\zeta$, any $a(p^{\star j})<1$ creates missing mass $(1-a(p^{\star j}))\cdot (p^{\star j})^{-s}\cdot Z'_p(s)$ that contradicts boundary saturation.
\end{remark}

\begin{remark}[Why signs/complex values break the method]
If $a(n)$ can be negative or complex, the comparison $Z(s)\le \zeta(s)$ can fail, so the mass argument no longer applies.

One might ask: could we recover the argument by taking absolute values? This does not work for two reasons. First, for complex-valued multiplicative functions, $|a(mn)| \ne |a(m)||a(n)|$ in general, so the Euler product structure is lost. Second, even for real-valued functions, taking absolute values changes the series: we would compare $\sum |a(n)|n^{-s}$ to $\zeta(s)$, but this tells us nothing about the original series $Z(s)$, which may have cancellations that alter its pole structure entirely. For instance, Dirichlet $L$-functions $L(s,\chi)$ with non-principal $\chi$ have no pole at $s=1$ despite $|\chi(n)|=1$ for $(n,q)=1$.
\end{remark}


\begin{remark}[Identifiability]\label{rem:identifiability}
The identity $Z(s)=\zeta(s)$ alone does not determine $\star$: any bijection $\phi$ of the $\star$-irreducibles with the primes, extended multiplicatively to $\N$, yields an operation $m\star_\phi n:=\phi^{-1}(\phi(m)\cdot\phi(n))$ with $Z(s)=\zeta(s)$ but potentially $\star_\phi\ne\cdot$. Our hypotheses (E1)--(E3) force $a\equiv 1$ and $Z=\zeta$, determining $\star$ up to the choice of bijection between $P$ and the primes. This is the precise residual freedom: the monoid structure is determined, but not its embedding into~$\N$.
\end{remark}

\section{Sharpness of the Hypotheses}\label{sec:sharpness}

The three hypotheses are necessary and logically independent.  We demonstrate this by exhibiting, for each hypothesis, a pair $(\star, A)$ satisfying the other two but violating the given one, with the conclusion of Theorem~\ref{thm:main} failing.

\subsection{Necessity of (E1): Euler Product}

(E1) is logically independent of (E2)--(E3): Example~\ref{ex:twisted} satisfies (E2) and (E3) but not (E1), yet the conclusion of Theorem~\ref{thm:main} still holds (since $a\equiv 1$ and $(\N,\star)\cong(\N,\cdot)$).  However, (E1) is also necessary: the following example satisfies (E2) and (E3) but not (E1), and the conclusion fails.

\begin{example}[Two-generator monoid]\label{ex:violate-E1-two-gen}
Let $\phi:\N_{\ge 0}^2\to\N\setminus\{0\}$ be a bijection satisfying $\phi(0,0)=1$, $\phi(k,0)=2^k$, and $\phi(0,k)=3^k$ for all $k\ge 1$.  (Such a bijection exists because $\N\setminus(\{2^a:a\ge 0\}\cup\{3^b:b\ge 0\})$ is countably infinite, so the remaining values of~$\phi$ can be assigned arbitrarily to the remaining pairs.)  Define $m\star n=\phi(\phi^{-1}(m)+\phi^{-1}(n))$.

Then $(\N,\star,1)\cong(\N_{\ge 0}^2,+,(0,0))$ is a free commutative monoid on two generators, with $\star$-irreducibles $P=\{2,3\}$, $\star$-powers $2^{\star k}=2^k$ and $3^{\star k}=3^k$, and unique $\star$-factorization.  The support is $A=\N$ with $a\equiv 1$, so $Z(s)=\zeta(s)$.

\emph{(E2) holds:}  $F_2(1)=1+\sum_{k\ge 1}2^{-k}=2<\infty$ and $F_3(1)=1+\sum_{k\ge 1}3^{-k}=3/2<\infty$.

\emph{(E3) holds:}  $\limsup_{s\to 1^+}(s-1)Z(s)=(s-1)\zeta(s)\to 1\ge 1$.

\emph{(E1) fails:}  $\prod_{p\in P}F_p(s)=F_2(s)\cdot F_3(s)=\sum_{a,b\ge 0}(2^a\cdot 3^b)^{-s}$, which sums only over $3$-smooth numbers, while $Z(s)=\zeta(s)$ sums over all positive integers.

\emph{Conclusion fails:}  The monoid $(\N,\star)$ has $|P|=2$ irreducibles while $(\N,\cdot)$ has infinitely many primes, so $(\N,\star)\not\cong(\N,\cdot)$.
\end{example}

Conceptually, both Example~\ref{ex:violate-E1-additive} (additive monoid, one generator) and Example~\ref{ex:violate-E1-two-gen} (two generators) have finitely many $\star$-irreducibles---the fundamental obstruction to isomorphism with $(\N,\cdot)$.  With one generator, a single irreducible must generate all of $\N\setminus\{1\}$, so $F_p(1)=\sum_{n\ge 2}1/n=\infty$ and (E2) automatically fails.  With two or more generators, each axis is a sparse subset of~$\N$, allowing (E2) to hold while (E1) still fails.

\subsection{Necessity of (E2): Local Convergence}

We first give a clean counterexample---a Boolean multiplicative monoid satisfying (E1) and (E3) but violating (E2)---showing that the conclusion of Theorem~\ref{thm:main} genuinely fails.

\begin{example}[Additive monoid]\label{ex:violate-E1-additive}
Define $n\star m = n+m-1$ for all $n,m\ge 1$.  This operation is associative, commutative, with identity $1$.  Its unique $\star$-irreducible is $2$: indeed $a\star b = a+b-1\ge 3$ for all $a,b\ge 2$, so no element other than $2$ is irreducible.  By induction $2^{\star k}=k+1$ for all $k\ge 1$, so every $n\ge 2$ factors uniquely as $n=2^{\star(n-1)}$.  In particular $(\N,\star)\cong(\N_{\ge 0},+)$ is the free commutative monoid on one generator.

Since every $n\ge 1$ is a $\star$-power of $2$, the support is $A=\N$ and $a\equiv 1$.  Axiom~(ii) and (E1) hold vacuously (the only coprime pairs under $\star$ involve $1$, so the Euler product has a single factor).  The Dirichlet series equals $Z(s)=\zeta(s)$, so (E3) holds.  However, the local Euler factor at the sole irreducible $2$ is
\[
F_2(1)=1+\sum_{k\ge 1}\frac{1}{k+1}=\infty,
\]
and (E2) fails.  Conclusion~(3) of Theorem~\ref{thm:main} also fails: $\star\ne\cdot$, since the monoid $(\N,\star)$ has one irreducible while $(\N,\cdot)$ has infinitely many primes.  Conceptually, without (E2) a single irreducible can generate all of $\N$, precluding the emergence of distinct primes.
\end{example}

The next two examples are informal illustrations---they display specific operations on $\star$-powers of a single irreducible rather than fully specified operations on all of $\N$---showing two mechanisms by which (E2) fails within the proof.

\begin{example}[Collapsing powers]\label{ex:violate-E2-diverge}
Suppose $p^{\star k} = p$ for all $k \ge 1$. Then the local Euler factor becomes
\[
F_p(s)=1+p^{-s}+p^{-s}+\cdots,
\]
which diverges for every $s>0$. Thus (E2) fails.  (This operation also violates unique factorization: $p = p\star p$ and cancellation give $1=p$, contradicting $p\ge 2$.)  Such operations cannot produce a well-defined Dirichlet series, let alone satisfy the rigidity conclusions.
\end{example}

\begin{example}[Sub-exponential growth]\label{ex:violate-E2-subexp}
Suppose $p^{\star k} = p\cdot k$ for all $k\ge 1$ (so $\star$-powers grow linearly rather than exponentially). Then
\[
F_p(1) = 1 + \sum_{k\ge 1}\frac{1}{pk} = 1 + \frac{1}{p}\sum_{k\ge 1}\frac{1}{k} = \infty,
\]
violating (E2). The sub-exponential growth of $\star$-powers prevents the local factor from converging, and the mass argument breaks down because the bound $L/F_p(1)$ in Step~4 of Proposition~\ref{prop:mass} becomes $0$ rather than a positive contradiction.
\end{example}

\subsection{Necessity of (E3): Boundary Mass Saturation}

We record two orthogonal ways (E3) can fail: deleting an entire prime family (prime-level sparsity) and deleting all prime powers (power-level sparsity). Both drop the boundary mass strictly below $1$.

\begin{example}[Missing an infinite family of irreducibles]\label{ex:violate-E3-missing}
With ordinary multiplication, let $A=\{1\}\cup\{n\ge3:\ n\ \text{odd}\}$ (exclude
all powers of $2$). Then
\[
Z(s)=\prod_{\substack{p\in\Primes\\ p\ \mathrm{odd}}}\,(1-p^{-s})^{-1}
= \zeta(s)\cdot(1-2^{-s}).
\]
Hence
\[
\lim_{s\to1^+}(s-1)Z(s)
= \lim_{s\to1^+}(s-1)\zeta(s)\cdot(1-2^{-s})
= 1\cdot\tfrac{1}{2}
= \frac{1}{2}\,<\,1,
\]
violating (E3). Here (E1) and (E2) hold ($F_p(1)=p/(p-1)<\infty$ for each odd prime) but the missing prime family reduces the boundary mass below the critical threshold.
\end{example}


\begin{example}[Square-free support]
From Example~\ref{ex:squarefree}, $Z(s) = \zeta(s)/\zeta(2s)$ has residue $6/\pi^2 \approx 0.608 < 1$ at $s=1$, violating (E3). Here (E1) and (E2) hold (each $F_p(1)=1+1/p<\infty$), but missing prime powers reduce the boundary mass below the threshold.
\end{example}




\section{Concluding Remarks}\label{sec:concluding}

We have shown that ordinary multiplication on $\N$ is uniquely characterized
(up to isomorphism) by three minimal analytic properties of its associated Dirichlet series:
an Euler product, local convergence of Euler factors, and boundary mass saturation.
Any Boolean multiplicative monoid satisfying (E1)--(E3) must have
$a\equiv 1$, $Z=\zeta$, and $(\N,\star)\cong(\N,\cdot)$.

The proof is a short mass argument: since $a(n)\le 1$ gives $Z\le\zeta$ termwise, any zero coefficient creates positive missing mass that contradicts boundary saturation. Unlike classical proofs in analytic number theory, it requires no Tauberian theorems, contour integration, or comparison of individual prime powers to ordinary powers.

The conclusion is sharp in two senses. First, all three hypotheses are necessary and logically independent (Section~\ref{sec:sharpness}). Second, the conclusion cannot be upgraded from isomorphism to identity: Remark~\ref{rem:psi-twist} exhibits an operation with $a\equiv 1$ and $Z=\zeta$ but $\star\ne\cdot$.

These two senses of sharpness reflect a deeper point: the analytic conditions (E1)--(E3) determine the multiplicative monoid structure completely---the underlying monoid $(\N,\star)$ is necessarily free commutative on countably many generators, isomorphic to $(\N,\cdot)$. But analytics alone cannot pin down the \emph{concrete realization} of this structure on $\N$: which specific integers serve as irreducibles, and in what order their $\star$-powers appear, remains undetermined. To force $\star = \cdot$ (not just $\star \cong \cdot$), one needs the non-analytic condition of order-compatibility (Corollary~\ref{cor:order-compat}).

Ordinary multiplication and the primes thus appear not as one
possibility among many but as the inevitable outcome of basic analytic
consistency---a form of \emph{rigidity} rather than global symmetry.

\paragraph{Open questions.}
Several natural directions emerge from this work:
\begin{enumerate}
\item \emph{Relaxing the coefficient constraint.}
The mass argument extends to $a(n) \in [0,1]$ (Remark~\ref{rem:beyond-boolean}). Can rigidity be established for bounded real coefficients, or does sign freedom destroy it?
\item \emph{From isomorphism to identity.}
Order-compatibility ($a\le b\Rightarrow a\star c\le b\star c$) is both necessary and sufficient for upgrading $\cong$ to $=$ (Corollary~\ref{cor:order-compat}).  Can the ordering of $\N$ be replaced by a weaker structural condition---for instance, one involving only the analytic behavior of individual Euler factors?
\item \emph{Characterizing other $L$-functions.}
Can minimal local-global conditions characterize Dirichlet $L$-functions or other members of the Selberg class?
\item \emph{Partial rigidity below the critical threshold.}
The threshold $1$ in (E3) is sharp: Examples~\ref{ex:violate-E3-missing} and~\ref{ex:squarefree} show that boundary mass strictly below $1$ admits non-classical structures. If $\limsup_{s\to 1^+}(s-1)Z(s)\ge 1-\varepsilon$ for small $\varepsilon>0$, does this force all but finitely many $\star$-irreducibles into the support?
\end{enumerate}

\paragraph{\textbf{Lean formalization.}}
A formalization of the algebraic and structural components of the proof in Lean~4 using the Mathlib library
is available at \url{https://github.com/eduardo-zambrano/rigidity}.
The formalization covers the Boolean multiplicative monoid (Definition~\ref{def:bmd}),
hypotheses (E1)--(E3), the full-support conclusion of Proposition~\ref{prop:mass},
and the isomorphism conclusion of Theorem~\ref{thm:main}.
Hypothesis~(E1) is axiomatized in its equivalent norm-compatibility form
(Lemma~\ref{lem:norm-compat}), since formalizing the Euler product identity
as an equality of infinite series and infinite products would require
substantial Mathlib infrastructure not yet available; hypotheses~(E2)
and~(E3) are formalized directly.
The algebraic steps---extending from irreducible powers to all $n\ge 1$
via unique $\star$-factorization, and assembling the monoid isomorphism---are
fully proved. The formalization cleanly separates the analytic from the
algebraic content. Three inputs are axiomatized rather than proved in
Mathlib: the analytic mass argument (Proposition~\ref{prop:mass}), the
infinitude of $\star$-irreducibles (deduced from the pole of $\zeta(s)$
via the Euler product), and the existence of the free-monoid isomorphism
(a standard algebraic fact). All files compile with
zero \texttt{sorry} blocks.




\bigskip
\noindent\textbf{Data availability.} No datasets were generated or analyzed.

\smallskip
\noindent\textbf{Conflict of interest.} The author declares no conflict of interest.


\begin{thebibliography}{99}

\bibitem{Beurling}
A. Beurling, \emph{Analyse de la loi asymptotique de la distribution des nombres premiers g\'en\'eralis\'es}, Acta Math. \textbf{68} (1937), 255--291.

\bibitem{Diamond1}
H. G. Diamond, \emph{The prime number theorem for Beurling's generalized numbers}, J. Number Theory \textbf{1} (1969), 200--207.

\bibitem{Diamond2}
H. G. Diamond, H. L. Montgomery, and U. M. A. Vorhauer, \emph{Beurling primes with large oscillation}, Math. Ann. \textbf{334} (2006), 1--36.

\bibitem{GHK}
A. Geroldinger and F. Halter-Koch, \emph{Non-Unique Factorizations: Algebraic, Combinatorial and Analytic Theory}, Pure and Applied Mathematics, vol.~278, Chapman \& Hall/CRC, Boca Raton, FL, 2006.

\bibitem{Hamburger}
H. Hamburger, \emph{\"Uber die Riemannsche Funktionalgleichung der $\zeta$-Funktion}, Math. Z. \textbf{10} (1921), 240--254.

\bibitem{Hecke}
E. Hecke, \emph{\"Uber die Bestimmung Dirichletscher Reihen durch ihre Funktionalgleichung}, Math. Ann. \textbf{112} (1936), 664--699.

\bibitem{Knopfmacher}
J. Knopfmacher, \emph{Abstract Analytic Number Theory}, 2nd ed., Dover Publications, New York, 1990.

\bibitem{Selberg}
A. Selberg, \emph{Old and new conjectures and results about a class of Dirichlet series}, in \emph{Proceedings of the Amalfi Conference on Analytic Number Theory}, Universit\`a di Salerno, 1992, pp. 367--385.

\bibitem{Tenenbaum}
G. Tenenbaum, \emph{Introduction to Analytic and Probabilistic Number Theory}, Cambridge Studies in Advanced Mathematics, vol. 46, Cambridge University Press, Cambridge, 1995.

\end{thebibliography}

\end{document}